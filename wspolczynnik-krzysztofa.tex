\documentclass{article}

% Language setting
% Replace `english' with e.g. `spanish' to change the document language
\usepackage[polish]{babel}

% Set page size and margins
% Replace `letterpaper' with `a4paper' for UK/EU standard size
\usepackage[letterpaper,top=2cm,bottom=2cm,left=4cm,right=4cm,marginparwidth=1.75cm]{geometry}

% Useful packages
\usepackage{amsmath}
\usepackage{graphicx}
\usepackage[colorlinks=true, allcolors=blue]{hyperref}

\usepackage{amsthm}
\theoremstyle{definition}
\newtheorem{definition}{Definicja}[section]

\title{Współczynnik Krzysztofa}

\author{
  Krzysztof Pobiarżyn\\
  \texttt{krzysztof@odra.dev}
  \and
  Maciej Zieliński\\
  \texttt{maciej@odra.dev}
  \and
  Jakub Płaskonka\\
  \texttt{kuba@odra.dev}
}

\begin{document}
\maketitle

\begin{abstract}
Praca ta po raz pierwszy podaje formalną definicję Współczynnika Krzysztofa. Jest on głównym i nieodłączynym już narzędziem podejmowania decyzji o akceptację, bądź odrzucenie nowego projektu.
\end{abstract}

\section{Motywacja}
Odpowiednia decyzja o podjęciu, bądź odrzuceniu projektu jest kluczowym elementem rozwoj małych zespołów. Błędna decyzja może doprowadzić do utraty rentowności zespółu, a przede wszystkim do spadku morale i utraty radości z budowania nowych projektów. Na pomoc przy podejmowaniu decyzji przychodzi miara jaką jest $ Współczynnik$ $Krzysztofa$ i zasada akceptowania $Kryterium$ $Krzysztofa$, które opisaliśmy w punkcie nr 2. Przykładowe zastosowanie przedstawiamy w punkcie nr 3. 

\section{Definicja}

$Współczynnik$ $Krzysztofa$ $K$ jest miarą opłacalności projektu. Zależy on od:

\begin{itemize}
\item przychodu z projektu $P$,
\item rzeczywistego kosztu projetu $R$.
\end{itemize}

\begin{definition}[Współczynnik Krzysztofa]
\[K = \frac{P}{\omega R}\]
\end{definition}

Gdzie $\omega$ to współczynnik trudności projektu.
Ekperymenty wykazały, że w odpowiednio długim ciągu opłacalnych i przyjemnych w realizacji projektów $\omega$ zmierza do $\pi$.

\begin{definition}[Kryterium Krzysztofa]
\[K \ge 1\]
\end{definition}

Projekt akceptowalny to taki, który spełnia $Kryterium$ $Krzysztofa$.

\section{Przykłady}
Dla przykładu przeprowadzimy analizę opłacalności projektu $A$. Cena projektu $A$ to 760 000 EUR, a jego koszt wykonania to 200 000 EUR.

\[K = \frac{760000}{\pi * 200000} = 1.2095775675 \ge 1\]

Tak więc projekt spełnia $Kryterium$ $Krzysztofa$, czyli nadaje się do wzięcia.
\end{document}
